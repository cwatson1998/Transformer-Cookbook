\chapter{Layer Normalization}
\label{sec:layernorm}

The original transformer model~\citep{vaswani-etal-2017-attention} used LayerNorm as an ``add \& norm'' operation after each sublayer, in order to normalize the activations after the residual connection following each sublayer.
Layer normalization~\citep{ba2016layer} is a normalization technique originally proposed for the purpose of reducing training time.
However, it has since been shown to have an impact on the expressiveness of the model, for example by affecting the Lipschitz continuity of the model~\citep{hahn-2020-theoretical} and the ability to express certain attention patterns~\citep{brody2023expressivity}.
In theoretical work, LayerNorm in some cases may complicate the proof, and in other cases may be an essential compoment of the proof.
In this section, we discuss the theoretical aspects of layer normalization, and how the literature on expressivity proofs has treated it.

\begin{definition}
    \AY{Will write}
\end{definition}

\section{Root Mean Square (RMS) Layer Norm}
RMS layer normalization~\citep{zhang2019root} is a normalization technique based on only re-scaling invariance, while ignoring re-centering invariance.
For a layer input \(x \in \mathbb{R}^{d}\), this is defined as follows:
\begin{equation}
    \mathrm{RMSNorm}(x) = \frac{x}{\mathrm{RMS}(x) + \epsilon},
    \quad \mathrm{RMS}(x) = \sqrt{\frac{1}{d} \sum_{i = 1}^{d} x_i ^2}
\end{equation}
where \(\epsilon > 0\) is a small number for numerical stability.
Note that when \(\mathbb{E}[x] = (x_1 + \cdots + x_d)/d = 0\), RMSNorm is equivalent to LayerNorm because \(\mathrm{RMS}(x) = \sqrt{\mathrm{Var}(x)}\).
While RMSNorm is performance-competitive with LayerNorm, its main advantage is computational efficiency, as one does not need to additionally compute \(\mathbb{E}[x]\).
RMSNorm is used in LLMs such as Llama~\citep{touvron2023llama}.

\section{Post-Norm vs Pre-Norm}

The original definition of the transformer used what is now known as a ``post-norm'' architecture, where layer normalization is applied after each residual connection. For instance:

\[LayerNorm(X+attention(X))\]

This is in contrast to the ``pre-norm'' architecture, where layer normalization is applied before each residual connection.

\[X+(attention(LayerNorm(X)))\]

The pre-norm architecture is currently adopted as the standard. This is because the post-norm architecture has problems with exploding gradients making training unstable, which the pre-norm architecture does not~\citep{xiong2020layer}. In terms of theoretical expressivity constructions, most proof try to circumvent layernorm. We note that \citet{hahn-rofin-2024-sensitive} and \citet{yang2024counting} both use post-norm.


% \AY{hahn+rofin sensitivity uses post-norm}

\section{Circumventing Layer Normalization}

\subsection{Zero mean}

Duplicate all dimensions, negate them, so that mean in each position is $0$, then LayerNorm only applies scaling factor

\LS{will look}

\section{Using Layer Normalization}

\subsection{$\varepsilon$ and Lipschitz-continuity}

Without LayerNorm or additional positional encodings, the model is Lipschitz continuous, as shown in \citet{hahn-2020-theoretical}, which subjects the model to limitations in expressivity. Adding LayerNorm with a small $\varepsilon$ can break this Lipschitz continuity, allowing the model to express more complex functions. This is used, for example, in \citet{yang2024counting} to ``blow up'' small values in the model - i.e.comparing count values $\frac{C}{i+1}$ which approach $0$ as $i$ increases.

\subsection{Normalize to \texorpdfstring{${\pm}1$}{plus/minus 1}}

It is possible to use LayerNorm to normalize every single activation to $\pm1$. This depends on all activations to have an absolute value of at least $|\varepsilon|$. Then, using \ref{sec:ffn_cpwl} we can construct a FFN that clips all activations to $\pm\varepsilon$. Applying LayerNorm after this FFN will normalize all activations to $\pm1$.

\subsection{Layer-Norm Hash} \label{sec:ln_hash}

Another way that LayerNorm can be used in theoretical constructions is to implement integer-key retrieval \citep{yao-2021-self-attention,merrill-sabharwal-2024-cot,merrill2024little}.
This idea is captured in a general mechanism by the \emph{layer-norm hash} mechanism proposed by \citet{merrill-sabharwal-2024-cot}.

\paragraph{Integer-Key Retrieval.}
Imagine we want to retrieve some value from previous tokens based on a key, where the key is an integer.
To do this with attention, we want a representation $\phi(i)$ for any integer $i$ such that
\begin{equation} \label{eq:integer-key-retrieval}
    \phi(i) \cdot \phi(j) = 1 \iff i = j .
\end{equation}
Then, for any integers $q_i, k_j \in \mathbb N$, saturated attention with query $\phi(q_i)$, key $\phi(k_j)$ and value $v_j$ will return an average of the $v_j$'s where $q_i = k_j$, allowing us to implement retrieval based on an integer key.

\paragraph{Layer-Norm Hash Implementation.}
As shown by \citet{merrill-sabharwal-2024-cot}, we can implement $\phi$ with layer-norm as follows. First, define
\begin{equation*}
    \phi(x, y) = LayerNorm(\langle x, y, -x, -y \rangle) .
\end{equation*}
Then, we can define the layer-norm hash of a single argument $x$ as
\begin{equation*}
    \phi(x) = \phi(x, 1)
\end{equation*}
This satisfies the desired property in \eqref{eq:integer-key-retrieval} and also satisfies the useful property that $\phi(x, y) = \phi(x/y) = \arctan(x/y)$.
\citet{merrill-sabharwal-2024-cot} use $\phi$ to simulate a Turing machine tape with chain-of-thought transformers: specifically, to retrieve the last value written to a previous index on the tape.
\citet{merrill2024little} use it to implement a binary tree construction with log-depth transformers.
More generally, it is useful in any construction that requires retrieval based on integer key matching.

\paragraph{Selective Pre-Norm.}
To make maximum use of the layer-norm hash, some papers assume a slight generalization of standard layernorm called \emph{masked} or \emph{selective} pre-norm \citep{merrill-sabharwal-2024-cot,merrill2024little}.
The problem without this extension is that the residual stream may store more than just a single integer, but standard pre-norm is applied to all values in the residual stream. Thus, rather than computing $\phi(x, 1)$, we will get $LayerNorm(\langle x, 1, \vec z \rangle)$, where $\vec z$ is some other information in the residual stream.
To circumvent this problem, masked pre-norm simply adds a linear transformation $W$ that gets multiplied by the residual stream before the layer-norm:

\[X + attention(LayerNorm(W X))\]

The linear transformation $W$ can mask out information encoded in the residual stream that is not relevant at this layer, allowing the network to compute the layer-norm hash of a specific value.
The model can also access more than one layer-norm hashed value in the same layer by using previous layers to normalize and store those values, and then using $W$ to select the indices where those values have been stored \citep{merrill-sabharwal-2024-cot}.
Finally, it is sufficient to let $W$ be diagonal, or equivalently, a vector that gets multiplied elementwise by $X$, to achieve these constructions.